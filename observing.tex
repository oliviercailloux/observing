\RequirePackage[l2tabu, orthodox]{nag}
\documentclass[version=3.21, pagesize, twoside=off, bibliography=totoc, DIV=calc, fontsize=12pt, a4paper]{scrartcl}
%Permits to copy eg x ⪰ y ⇔ v(x) ≥ v(y) from PDF to unicode data, and to search. From pdfTeX users manual. See https://tex.stackexchange.com/posts/comments/1203887.
	\input glyphtounicode
	\pdfgentounicode=1
%Latin Modern has more glyphs than Computer Modern, such as diacritical characters. fntguide commands to load the font before fontenc, to prevent default loading of cmr.
	\usepackage{lmodern}
%Encode resulting accented characters correctly in resulting PDF, permits copy from PDF.
	\usepackage[T1]{fontenc}
%UTF8 seems to be the default in recent TeX installations, but not all, see https://tex.stackexchange.com/a/370280.
	\usepackage[utf8]{inputenc}
%Provides \newunicodechar for easy definition of supplementary UTF8 characters such as → or ≤ for use in source code.
	\usepackage{newunicodechar}
%Text Companion fonts, much used together with CM-like fonts. Provides \texteuro and commands for text mode characters such as \textminus, \textrightarrow, \textlbrackdbl.
	\usepackage{textcomp}
%St Mary’s Road symbol font, used for ⟦ = \llbracket.
	%\usepackage{stmaryrd}
%Solves bug in lmodern, https://tex.stackexchange.com/a/261188; probably useful only for unusually big font sizes; and probably better to use exscale instead. Note that the authors of exscale write against this trick.
	%\DeclareFontShape{OMX}{cmex}{m}{n}{
		%<-7.5> cmex7
		%<7.5-8.5> cmex8
		%<8.5-9.5> cmex9
		%<9.5-> cmex10
	%}{}
	%\SetSymbolFont{largesymbols}{normal}{OMX}{cmex}{m}{n}
%More symbols (such as \sum) available in bold version, see https://github.com/latex3/latex2e/issues/71.
	\DeclareFontShape{OMX}{cmex}{bx}{n}{%
	   <->sfixed*cmexb10%
	   }{}
	\SetSymbolFont{largesymbols}{bold}{OMX}{cmex}{bx}{n}
%For small caps also in italics, see https://tex.stackexchange.com/questions/32942/italic-shape-needed-in-small-caps-fonts, https://tex.stackexchange.com/questions/284338/italic-small-caps-not-working.
	\usepackage{slantsc}
	\AtBeginDocument{%
		%“Since nearly no font family will contain real italic small caps variants, the best approach is to substitute them by slanted variants.” -- slantsc doc
		%\DeclareFontShape{T1}{lmr}{m}{scit}{<->ssub*lmr/m/scsl}{}%
		%There’s no bold small caps in Latin Modern, we switch to Computer Modern for bold small caps, see https://tex.stackexchange.com/a/22241
		%\DeclareFontShape{T1}{lmr}{bx}{sc}{<->ssub*cmr/bx/sc}{}%
		%\DeclareFontShape{T1}{lmr}{bx}{scit}{<->ssub*cmr/bx/scsl}{}%
	}
%Warn about missing characters.
	\tracinglostchars=2
%Nicer tables: provides \toprule, \midrule, \bottomrule.
	%\usepackage{booktabs}
%For new column type X which stretches; can be used together with booktabs, see https://tex.stackexchange.com/a/97137. “tabularx modifies the widths of the columns, whereas tabular* modifies the widths of the inter-column spaces.” Loads array.
	%\usepackage{tabularx}
%math-mode version of "l" column type. Requires \usepackage{array}.
	%\usepackage{array}
	%\newcolumntype{L}{>{$}l<{$}}
%Provides \xpretocmd and loads etoolbox which provides \apptocmd, \patchcmd, \newtoggle… Also loads xparse, which provides \NewDocumentCommand and similar commands intended as replacement of \newcommand in LaTeX3 for defining commands (see https://tex.stackexchange.com/q/98152 and https://github.com/latex3/latex2e/issues/89).
	\usepackage{xpatch}
%ntheorem doc says: “empheq provides an enhanced vertical placement of the endmarks”; must be loaded before ntheorem. Loads the mathtools package, which loads and fixes some bugs in amsmath and provides \DeclarePairedDelimiter. amsmath is considered a basic, mandatory package nowadays (Grätzer, More Math Into LaTeX).
	\usepackage[ntheorem]{empheq}
%Package frenchb asks to load natbib before babel-french. Package hyperref asks to load natbib before hyperref.
	\usepackage{natbib}

\newtoggle{LCpres}
	\newtoggle{LCart}
	\newtoggle{LCposter}
	\makeatletter
	\@ifclassloaded{beamer}{
		\toggletrue{LCpres}
		\togglefalse{LCart}
		\togglefalse{LCposter}
		\wlog{Presentation mode}
	}{
		\@ifclassloaded{tikzposter}{
			\toggletrue{LCposter}
			\togglefalse{LCpres}
			\togglefalse{LCart}
			\wlog{Poster mode}
		}{
			\toggletrue{LCart}
			\togglefalse{LCpres}
			\togglefalse{LCposter}
			\wlog{Article mode}
		}
	}
	\makeatother%

%Language options ([french, english]) should be on the document level (last is main); except with tikzposter: put [french, english] options next to \usepackage{babel} to avoid warning. beamer uses the \translate command for the appendix: omitting babel results in a warning, see https://github.com/josephwright/beamer/issues/449. Babel also seems required for \refname.
	\iftoggle{LCpres}{
		\usepackage{babel}
	}{
	}
	%\frenchbsetup{AutoSpacePunctuation=false}
%listings (1.7) does not allow multi-byte encodings. listingsutf8 works around this only for characters that can be represented in a known one-byte encoding and only for \lstinputlisting. Other workarounds: use literate mechanism; or escape to LaTeX (but breaks alignment).
	%\usepackage{listings}
	%\lstset{tabsize=2, basicstyle=\ttfamily, escapechar=§, literate={é}{{\'e}}1}
%I favor acro over acronym because the former is more recently updated (2018 VS 2015 at time of writing); has a longer user manual (about 40 pages VS 6 pages if not counting the example and implementation parts); has a command for capitalization; and acronym suffers a nasty bug when ac used in section, see https://tex.stackexchange.com/q/103483 (though this might be the fault of the silence package and might be solved in more recent versions, I do not know) and from a bug when used with cleveref, see https://tex.stackexchange.com/q/71364. However, loading it makes compilation time (one pass on this template) go from 0.6 to 1.4 seconds, see https://bitbucket.org/cgnieder/acro/issues/115. Option short-format not usable in the package options as it is fragile, see https://tex.stackexchange.com/q/466882.
	%\usepackage[single]{acro}
	%\acsetup{short-format = {\scshape}}
	%\DeclareAcronym{AMCD}{short=amcd, long={Aide Multicritère à la Décision}}
\DeclareAcronym{AR}{short=ar, long={Argumentative Recommender}}
\DeclareAcronym{DA}{short=da, long={Decision Analysis}}
\DeclareAcronym{DJ}{short=dj, long={Deliberated Judgment}}
\DeclareAcronym{DM}{short=dm, long={Decision Maker}}
\DeclareAcronym{DP}{short=dp, long={Deliberated Preference}}
\DeclareAcronym{MAVT}{short=mavt, long={Multiple Attribute Value Theory}}
\DeclareAcronym{MCDA}{short=mcda, long={Multicriteria Decision Aid}}
\DeclareAcronym{MIP}{short=mip, long={Mixed Integer Program}}


\iftoggle{LCpres}{
	%I favor fmtcount over nth because it is loaded by datetime anyway; and fmtcount warns about possible conflicts when loaded after nth.
	\usepackage{fmtcount}
	%For nice input of date of presentation. Must be loaded after the babel package. Has possible problems with srcletter: https://golatex.de/verwendung-von-babel-und-datetime-in-scrlttr2-schlaegt-fehlt-t14779.html.
	\usepackage[nodayofweek]{datetime}
}{
}
%For presentations, Beamer implicitely uses the pdfusetitle option. ntheorem doc says to load hyperref “before the first use of \newtheorem”. autonum doc mandates option hypertexnames=false. I want to highlight links only if necessary for the reader to recognize it as a link, to reduce distraction. In presentations, this is already taken care of by beamer (https://tex.stackexchange.com/a/262014). If using colorlinks=true in a presentation, see https://tex.stackexchange.com/q/203056. Crashes the first compilation with tikzposter, just compile again and the problem disappears, see https://tex.stackexchange.com/q/254257.
\makeatletter
\iftoggle{LCpres}{
	\usepackage{hyperref}
}{
	\usepackage[hypertexnames=false, pdfusetitle, linkbordercolor={1 1 1}, citebordercolor={1 1 1}, urlbordercolor={1 1 1}]{hyperref}
	%https://tex.stackexchange.com/a/466235
	\pdfstringdefDisableCommands{%
		\let\thanks\@gobble
	}
}
\makeatother
%urlbordercolor is used both for \url and \doi, which I think shouldn’t be colored, and for \href, thus might want to color manually when required. Requires xcolor.
	\NewDocumentCommand{\hrefblue}{mm}{\textcolor{blue}{\href{#1}{#2}}}
%hyperref doc says: “Package bookmark replaces hyperref’s bookmark organization by a new algorithm (...) Therefore I recommend using this package”.
	\usepackage{bookmark}
%Need to invoke hyperref explicitly to link to line numbers: \hyperlink{lintarget:mylinelabel}{\ref*{lin:mylinelabel}}, with \ref* to disable automatic link. Also see https://tex.stackexchange.com/q/428656 for referencing lines from another document.
	%\usepackage{lineno}
	%\NewDocumentCommand{\llabel}{m}{\hypertarget{lintarget:#1}{}\linelabel{lin:#1}}
	%\setlength\linenumbersep{9mm}
%For complex authors blocks. Seems like authblk wants to be later than hyperref, but sooner than silence. See https://tex.stackexchange.com/q/475513 for the patch to hyperref pdfauthor.
	\ExplSyntaxOn
	\seq_new:N \g_oc_hrauthor_seq
	\NewDocumentCommand{\addhrauthor}{m}{
		\seq_gput_right:Nn \g_oc_hrauthor_seq { #1 }
	}
	%Should be \NewExpandableDocumentCommand, but this is not yet provided by my version of xparse
	\DeclareExpandableDocumentCommand{\hrauthor}{}{
		\seq_use:Nn \g_oc_hrauthor_seq {,~}
	}
	\ExplSyntaxOff
	{
		\catcode`#=11\relax
		\gdef\fixauthor{\xpretocmd{\author}{\addhrauthor{#2}}{}{}}%
	}
	\iftoggle{LCart}{
		\usepackage{authblk}
		\renewcommand\Affilfont{\small}
		\fixauthor
		\AtBeginDocument{
		    \hypersetup{pdfauthor={\hrauthor}}
		}
	}{
	}
%I do not use floatrow, because it requires an ugly hack for proper functioning with KOMA script (see scrhack doc). Instead, the following command centers all floats (using \centering, as the center environment adds space, http://texblog.net/latex-archive/layout/center-centering/), and I manually place my table captions above and figure captions below their contents (https://tex.stackexchange.com/a/3253).
	\makeatletter
	\g@addto@macro\@floatboxreset\centering
	\makeatother
%Permits to customize enumeration display and references
	%\nottoggle{LCpres}{
		%\usepackage{enumitem} %follow list environments by a string to customize enumeration, example: \begin{description}[itemindent=8em, labelwidth=!] or \begin{enumerate}[label=({\roman*}), ref={\roman*}].
	%}{
	%}
%Provides \Centering, \RaggedLeft, and \RaggedRight and environments Center, FlushLeft, and FlushRight, which allow hyphenation. With tikzposter, seems to cause 1=1 to be printed in the middle of the poster.
	%\usepackage{ragged2e}
%To typeset units by closely following the “official” rules.
	%\usepackage[strict]{siunitx}
%Turns the doi provided by some bibliography styles into URLs. However, uses old-style dx.doi url (see 3.8 DOI system Proxy Server technical details, “Users may resolve DOI names that are structured to use the DOI system Proxy Server (https://doi.org (current, preferred) or earlier syntax http://dx.doi.org).”, https://www.doi.org/doi_handbook/3_Resolution.html). The patch solves this.
	\usepackage{doi}
	\makeatletter
	\patchcmd{\@doi}{http://dx.doi.org}{https://doi.org}{}{}
	\makeatother
%Makes sure upper case greek letters are italic as well.
	\usepackage{fixmath}
%Provides \mathbb; obsoletes latexsym (see http://tug.ctan.org/macros/latex/base/latexsym.dtx). Relatedly, \usepackage{eucal} to change the mathcal font and \usepackage[mathscr]{eucal} (apparently equivalent to \usepackage[mathscr]{euscript}) to supplement \mathcal with \mathscr. This last option is not very useful as both fonts are similar, and the intent of the authors of eucal was to provide a replacement to mathcal (see doc euscript). Also provides \mathfrak for supplementary letters.
	\usepackage{amsfonts}
%Provides a beautiful (IMHO) \mathscr and really different than \mathcal, for supplementary uppercase letters. But there is no bold version. Alternative: mathrsfs (more slanted), but when used with tikzposter, it warns about size substitution, see https://tex.stackexchange.com/q/495167.
	\usepackage[scr]{rsfso}
%Multiple means to produce bold math: \mathbf, \boldmath (defined to be \mathversion{bold}, see fntguide), \pmb, \boldsymbol (all legacy, from LaTeX base and AMS), \bm (the most recommended one), \mathbold from package fixmath (I don’t see its advantage over \boldsymbol).
%“The \boldsymbol command is obtained preferably by using the bm package, which provides a newer, more powerful version than the one provided by the amsmath package. Generally speaking, it is ill-advised to apply \boldsymbol to more than one symbol at a time.” — AMS Short math guide. “If no bold font appears to be available for a particular symbol, \bm will use ‘poor man’s bold’” — bm. It is “best to load the package after any packages that define new symbol fonts” – bm. bm defines \boldsymbol as synonym to \bm. \boldmath accesses the correct font if it exists; it is used by \bm when appropriate. See https://tex.stackexchange.com/a/10643 and https://github.com/latex3/latex2e/issues/71 for some difficulties with \bm.
	\usepackage{bm}
	\nottoggle{LCpres}{
	%https://ctan.org/pkg/amsmath recommends ntheorem, which supersedes amsthm, which corrects the spacing of proclamations and allows for theoremstyle. Option standard loads amssymb and latexsym. Must be loaded after amsmath (from ntheorem doc). From cleveref doc, “ntheorem is fully supported and even recommended”; says to load cleveref after ntheorem. When used with tikzposter, warns about size substitution for the lasy (latexsym) font when using \url, because ntheorem loads latexsym; relatedly (but not directly related to ntheorem), size substitution warning with the cmex font happens when loading amsmath and using \url. According to https://tex.stackexchange.com/q/535950, ntheorem “seems essentially unmaintaned and has severe problems”, but I use it anyway because it is very handy. Yields “! LaTeX Error: Something's wrong--perhaps a missing \item.” if some theorem follows thebibliography.
		\usepackage[thmmarks, amsmath, standard, hyperref]{ntheorem}
		%empheq doc says to do this after loading ntheorem
		\usetagform{default}
	%Provides \cref. Unfortunately, cref fails when the language is French and referring to a label whose name contains a colon (https://tex.stackexchange.com/q/83798). Use \cref{sec\string:intro} to work around this. cleveref should go “laster” than hyperref.
		\usepackage{cleveref}
	}{
	}
	\nottoggle{LCposter}{
	%Equations get numbers iff they are referenced. Loading order should be “amsmath → hyperref → cleveref → autonum”, according to autonum doc. Use this in preference to the showonlyrefs option from mathtools, see https://tex.stackexchange.com/q/459918 and autonum doc. See https://tex.stackexchange.com/a/285953 for the etex line. Incompatible with my version of tikzposter (produces “! Improper \prevdepth”).
		\expandafter\def\csname ver@etex.sty\endcsname{3000/12/31}\let\globcount\newcount
		\usepackage{autonum}
	}{
	}
%Also loaded by tikz.
	\usepackage{xcolor}
\iftoggle{LCpres}{
	\usepackage{tikz}
	%\usetikzlibrary{babel, matrix, fit, plotmarks, calc, trees, shapes.geometric, positioning, plothandlers, arrows, shapes.multipart}
}{
}
%Vizualization, on top of TikZ
	%\usepackage{pgfplots}
	%\pgfplotsset{compat=1.14}
\usepackage{graphicx}
	\graphicspath{{graphics/}}

%Provides \printlength{length}, useful for debugging.
	%\usepackage{printlen}
	%\uselengthunit{mm}

\iftoggle{LCpres}{
	\usepackage{appendixnumberbeamer}
	%I have yet to see anyone actually use these navigation symbols; let’s disable them
	\setbeamertemplate{navigation symbols}{} 
	\usepackage{preamble/beamerthemeParisFrance}
	\setcounter{tocdepth}{10}
}{
}

%Do not use the displaymath environment: use equation. Do not use the eqnarray or eqnarray* environments: use align(*). This improves spacing. (See l2tabu or amsldoc.)


%Requires package xcolor.
\NewDocumentCommand{\commentOC}{m}{\textcolor{blue}{\small$\big[$OC: #1$\big]$}}
%Requires package babel and option [french]. According to babel doc, need two braces around \selectlanguage to make the changes really local.
\NewDocumentCommand{\commentOCf}{m}{\textcolor{blue}{{\small\selectlanguage{french}$\big[$OC : #1$\big]$}}}
\NewDocumentCommand{\commentRS}{m}{\textcolor{red}{\small$\big[$RS: #1$\big]$}}
\NewDocumentCommand{\commentRSf}{m}{\textcolor{red}{{\small\selectlanguage{french}$\big[$RS : #1$\big]$}}}

\bibliographystyle{abbrvnat}
\NewDocumentCommand{\possessivecite}{mO{}}{\citeauthor{#1}’s \citeyearpar[#2]{#1}}

%https://tex.stackexchange.com/a/467188, https://tex.stackexchange.com/a/36088 - uncomment if one of those symbols is used.
%\DeclareFontFamily{U} {MnSymbolD}{}
%\DeclareFontShape{U}{MnSymbolD}{m}{n}{
%  <-6> MnSymbolD5
%  <6-7> MnSymbolD6
%  <7-8> MnSymbolD7
%  <8-9> MnSymbolD8
%  <9-10> MnSymbolD9
%  <10-12> MnSymbolD10
%  <12-> MnSymbolD12}{}
%\DeclareFontShape{U}{MnSymbolD}{b}{n}{
%  <-6> MnSymbolD-Bold5
%  <6-7> MnSymbolD-Bold6
%  <7-8> MnSymbolD-Bold7
%  <8-9> MnSymbolD-Bold8
%  <9-10> MnSymbolD-Bold9
%  <10-12> MnSymbolD-Bold10
%  <12-> MnSymbolD-Bold12}{}
%\DeclareSymbolFont{MnSyD} {U} {MnSymbolD}{m}{n}
%\DeclareMathSymbol{\ntriplesim}{\mathrel}{MnSyD}{126}
%\DeclareMathSymbol{\nlessgtr}{\mathrel}{MnSyD}{192}
%\DeclareMathSymbol{\ngtrless}{\mathrel}{MnSyD}{193}
%\DeclareMathSymbol{\nlesseqgtr}{\mathrel}{MnSyD}{194}
%\DeclareMathSymbol{\ngtreqless}{\mathrel}{MnSyD}{195}
%\DeclareMathSymbol{\nlesseqgtrslant}{\mathrel}{MnSyD}{198}
%\DeclareMathSymbol{\ngtreqlessslant}{\mathrel}{MnSyD}{199}
%\DeclareMathSymbol{\npreccurlyeq}{\mathrel}{MnSyD}{228}
%\DeclareMathSymbol{\nsucccurlyeq}{\mathrel}{MnSyD}{229}
%\DeclareFontFamily{U} {MnSymbolA}{}
%\DeclareFontShape{U}{MnSymbolA}{m}{n}{
%  <-6> MnSymbolA5
%  <6-7> MnSymbolA6
%  <7-8> MnSymbolA7
%  <8-9> MnSymbolA8
%  <9-10> MnSymbolA9
%  <10-12> MnSymbolA10
%  <12-> MnSymbolA12}{}
%\DeclareFontShape{U}{MnSymbolA}{b}{n}{
%  <-6> MnSymbolA-Bold5
%  <6-7> MnSymbolA-Bold6
%  <7-8> MnSymbolA-Bold7
%  <8-9> MnSymbolA-Bold8
%  <9-10> MnSymbolA-Bold9
%  <10-12> MnSymbolA-Bold10
%  <12-> MnSymbolA-Bold12}{}
%\DeclareSymbolFont{MnSyA} {U} {MnSymbolA}{m}{n}
%%Rightwards wave arrow: ↝. Alternative: \rightsquigarrow from amssymb, but it’s uglier
%\DeclareMathSymbol{\rightlsquigarrow}{\mathrel}{MnSyA}{160}

%03B3 Greek Small Letter Gamma
\newunicodechar{γ}{\gamma}
%03B4 Greek Small Letter Delta
\newunicodechar{δ}{\delta}
%2115 Double-Struck Capital N
\newunicodechar{ℕ}{\mathbb{N}}
%211D Double-Struck Capital R
\newunicodechar{ℝ}{\mathbb{R}}
%21CF Rightwards Double Arrow with Stroke
\newunicodechar{⇏}{\nRightarrow}
%21D2 Rightwards Double Arrow
\newunicodechar{⇒}{\ensuremath{\Rightarrow}}
%21D4 Left Right Double Arrow
\newunicodechar{⇔}{\Leftrightarrow}
%21DD Rightwards Squiggle Arrow
\newunicodechar{⇝}{\rightsquigarrow}
%2205 Empty Set
\newunicodechar{∅}{\emptyset}
%2212 Minus Sign
\newunicodechar{−}{\ifmmode{-}\else\textminus\fi}
%2227 Logical And
\newunicodechar{∧}{\land}
%2228 Logical Or
\newunicodechar{∨}{\lor}
%2229 Intersection
\newunicodechar{∩}{\cap}
%222A Union
\newunicodechar{∪}{\cup}
%2260 Not Equal To (handy also as text in informal writing)
\newunicodechar{≠}{\ensuremath{\neq}}
%2264 Less-Than or Equal To
\newunicodechar{≤}{\leq}
%2265 Greater-Than or Equal To
\newunicodechar{≥}{\geq}
%2270 Neither Less-Than nor Equal To
\newunicodechar{≰}{\nleq}
%2271 Neither Greater-Than nor Equal To
\newunicodechar{≱}{\ngeq}
%2272 Less-Than or Equivalent To
\newunicodechar{≲}{\lesssim}
%2273 Greater-Than or Equivalent To
\newunicodechar{≳}{\gtrsim}
%2274 Neither Less-Than nor Equivalent To – also, from MnSymbol: \nprecsim, a more exact match to the Unicode symbol; and \npreccurlyeq, too small
\newunicodechar{≴}{\not\preccurlyeq}
%2275 Neither Greater-Than nor Equivalent To
\newunicodechar{≵}{\not\succcurlyeq}
%2279 Neither Greater-Than nor Less-Than – requires MnSymbol; also \nlessgtr from txfonts/pxfonts, \ngtreqless from MnSymbol (but much higher), \ngtrless from MnSymbol (a more exact match to the Unicode symbol); for incomparability (not matching this Unicode symbol), may also consider \ntriplesim from MnSymbol,\nparallelslant from fourier, \between from mathabx, or ⋈
\newunicodechar{≹}{\ngtreqlessslant}
%227A Precedes
\newunicodechar{≺}{\prec}
%227B Succeeds
\newunicodechar{≻}{\succ}
%227C Precedes or Equal To
\newunicodechar{≼}{\preccurlyeq}
%227D Succeeds or Equal To
\newunicodechar{≽}{\succcurlyeq}
%227E Precedes or Equivalent To
\newunicodechar{≾}{\precsim}
%227F Succeeds or Equivalent To
\newunicodechar{≿}{\succsim}
%2280 Does Not Precede
\newunicodechar{⊀}{\nprec}
%2281 Does Not Succeed
\newunicodechar{⊁}{\nsucc}
%22B2 Normal Subgroup Of – using \vartriangleleft from amsfonts, which goes well with \trianglelefteq, \ntriangleright, and so on, also from amsfonts; another possibility is \lhd from latexsym, which seems visually equivalent to \vartriangleleft from amsfonts; latexsym also has ⊴=\unlhd, but doesn’t have a symbol for ⊴. Other related symbols: \triangleleft from latesym package is too small; fdsymbol provides \triangleleft=\medtriangleleft and \vartriangleleft=\smalltriangleleft; MnSymbol provides \medtriangleleft and \vartriangleleft=\lessclosed=\lhd which are smaller than \vartriangleleft from amsfont; \vartriangleleft from mathabx (p. 67), looks different (wider); also \vartriangleleft from boisik (p. 69) looks still different; \vartriangleleft=\lhd from stix are smaller. Oddly enough, \triangleright appears as the LMMathItalic12-Regular font whereas \rhd appears as LASY10 and \vartriangleright appears as MSAM10.
\newunicodechar{⊲}{\vartriangleleft}
%22B3 Contains as Normal Subgroup (also: 25B7 White right-pointing triangle or 25B9 White right-pointing small triangle)
\newunicodechar{⊳}{\vartriangleright}
%22B4 Normal Subgroup of or Equal To
\newunicodechar{⊴}{\trianglelefteq}
%22B5 Contains as Normal Subgroup or Equal To
\newunicodechar{⊵}{\trianglerighteq}
%22C8 Bowtie
\newunicodechar{⋈}{\bowtie}
%22EA Not Normal Subgroup Of
\newunicodechar{⋪}{\ntriangleleft}
%22EB Does Not Contain As Normal Subgroup
\newunicodechar{⋫}{\ntriangleright}
%22EC Not Normal Subgroup of or Equal To
\newunicodechar{⋬}{\ntrianglelefteq}
%22ED Does Not Contain as Normal Subgroup or Equal
\newunicodechar{⋭}{\ntrianglerighteq}
%25A1 White Square
\newunicodechar{□}{\Box}
%27E6 Mathematical Left White Square Bracket – requires stmaryrd (alternative: \text{\textlbrackdbl}, but ugly if used in an italicized text such as a theorem)
\newunicodechar{⟦}{\llbracket}
%27E7 Mathematical Right White Square Bracket
\newunicodechar{⟧}{\rrbracket}
%27FC Long Rightwards Arrow from Bar
\newunicodechar{⟼}{\longmapsto}
%2AB0 Succeeds Above Single-Line Equals Sign
\newunicodechar{⪰}{\succeq}
%301A Left White Square Bracket
\newunicodechar{〚}{\textlbrackdbl}
%301B Right White Square Bracket
\newunicodechar{〛}{\textrbrackdbl}
%→ is defined by default as \textrightarrow, which is invalid in math mode. Same thing for the three other commands. Using \DeclareUnicodeCharacter instead of \newunicodechar because the latter warns about the previous definition.
%→ Rightwards Arrow
\DeclareUnicodeCharacter{2192}{\ifmmode\rightarrow\else\textrightarrow\fi}
%¬ Not Sign
\DeclareUnicodeCharacter{00AC}{\ifmmode\lnot\else\textlnot\fi}
%… Horizontal Ellipsis
\DeclareUnicodeCharacter{2026}{\ifmmode\dots\else\textellipsis\fi}
%× Multiplication Sign
\DeclareUnicodeCharacter{00D7}{\ifmmode\times\else\texttimes\fi}
%Permits to really obtain a straight quote when typing a straight quote; potentially dangerous, see https://tex.stackexchange.com/a/521999
\catcode`\'=\active
\DeclareUnicodeCharacter{0027}{\ifmmode^\prime\else\textquotesingle\fi}


\NewDocumentCommand{\R}{}{ℝ}
\NewDocumentCommand{\N}{}{ℕ}
%\mathscr is rounder than \mathcal.
\NewDocumentCommand{\powerset}{m}{\mathscr{P}(#1)}
%Powerset without zero.
\NewDocumentCommand{\powersetz}{m}{\mathscr{P}^*(#1)}
%https://tex.stackexchange.com/a/45732, works within both \set and \set*, same spacing than \mid (https://tex.stackexchange.com/a/52905).
\NewDocumentCommand{\suchthat}{}{\;\ifnum\currentgrouptype=16 \middle\fi|\;}
%Integer interval.
\NewDocumentCommand{\intvl}{m}{⟦#1⟧}
%Allows for \abs and \abs*, which resizes the delimiters.
\DeclarePairedDelimiter\abs{\lvert}{\rvert}
\DeclarePairedDelimiter\card{\lvert}{\rvert}
%Perhaps should use U+2016 ‖ DOUBLE VERTICAL LINE here?
\DeclarePairedDelimiter\norm{\lVert}{\rVert}
%From mathtools. Better than using the package braket because braket introduces possibly undesirable space. Then: \begin{equation}\set*{x \in \R^2 \suchthat \norm{x}<5}\end{equation}.
\DeclarePairedDelimiter\set{\{}{\}}
\DeclareMathOperator*{\argmax}{arg\,max}
\DeclareMathOperator*{\argmin}{arg\,min}

%UTR #25: Unicode support for mathematics recommend to use the straight form of phi (by default, given by \phi) rather than the curly one (by default, given by \varphi), and thus use \phi for the mathematical symbol and not \varphi. I however prefer the curly form because the straight form is too easy to mix up with the symbol for empty set.
\let\phi\varphi

%The amssymb solution.
%\NewDocumentCommand{\restr}{mm}{{#1}_{\restriction #2}}
%Another acceptable solution.
%\NewDocumentCommand{\restr}{mm}{{#1|}_{#2}}
%https://tex.stackexchange.com/a/278631; drawback being that sometimes the text collides with the line below.
\NewDocumentCommand\restr{mm}{#1\raisebox{-.5ex}{$|$}_{#2}}


%Decision Theory (MCDA and SC)
\NewDocumentCommand{\allalts}{}{\mathscr{A}}
\NewDocumentCommand{\allcrits}{}{\mathscr{C}}
\NewDocumentCommand{\alts}{}{A}
\NewDocumentCommand{\dm}{}{i}
\NewDocumentCommand{\allF}{}{\mathscr{F}}
\NewDocumentCommand{\allvoters}{}{\mathscr{N}}
\NewDocumentCommand{\voters}{}{N}
\NewDocumentCommand{\allprofs}{}{\boldsymbol{\mathcal{R}}}
\NewDocumentCommand{\prof}{}{\boldsymbol{R}}
\NewDocumentCommand{\linors}{}{\mathscr{L}(\allalts)}
%Thanks to https://tex.stackexchange.com/q/154549
	%\makeatletter
	%\def\@myRgood@#1#2{\mathrel{R^X_{#2}}}
	%\def\myRgood{\@ifnextchar_{\@myRgood@}{\mathrel{R^X}}}
	%\makeatother

%Deliberated Judgment
\NewDocumentCommand{\allargs}{}{S^*}
\NewDocumentCommand{\args}{}{S}
\NewDocumentCommand{\ar}{}{s}
\NewDocumentCommand{\allprops}{}{T}
\NewDocumentCommand{\prop}{}{t}
\NewDocumentCommand{\ileadsto}{}{⇝}
\NewDocumentCommand{\ibeatse}{}{⊳_\exists}
\NewDocumentCommand{\nibeatse}{}{⋫_\exists}
\NewDocumentCommand{\ibeatsst}{}{⊳_\forall}
\NewDocumentCommand{\nibeatsst}{}{⋫_\forall}
\NewDocumentCommand{\mleadsto}{O{\eta}}{⇝_{#1}}
\NewDocumentCommand{\mbeats}{O{\eta}}{⊳_{#1}}
\NewDocumentCommand{\ibeatseinv}{}{⊳_\exists^{-1}}

\NewDocumentCommand{\phibar}{}{\overline{\phi}}
\NewDocumentCommand{\Astar}{}{A^*}
\NewDocumentCommand{\pib}{}{\bm{\pi}}
\NewDocumentCommand{\piQ}{}{\restr{\pi}{Q}}
\NewDocumentCommand{\pibQ}{}{\restr{\bm{\pi}}{Q}}
\NewDocumentCommand{\picorrpibQ}{}{\set{\pi \suchthat \restr{\pi}{Q} = \restr{\bm{\pi}}{Q}}}
\NewDocumentCommand{\picorrpiQ}{}{\set{\pi' \suchthat \restr{\pi'}{Q} = \restr{\pi}{Q}}}
\NewDocumentCommand{\cpiQ}{}{c\big(\piQ\big)}
\NewDocumentCommand{\cpibQ}{}{c\big(\pibQ\big)}

%Logic
\NewDocumentCommand{\ltru}{}{\texttt{T}}
\NewDocumentCommand{\lfal}{}{\texttt{F}}



%I find these settings useful in draft mode. Should be removed for final versions.
	%Which line breaks are chosen: accept worse lines, therefore reducing risk of overfull lines. Default = 200.
		\tolerance=2000
	%Accept overfull hbox up to...
		\hfuzz=2cm
	%Reduces verbosity about the bad line breaks.
		\hbadness 5000
	%Reduces verbosity about the underful vboxes.
		\vbadness=1300

\title{Observing deliberated judgments \thanks{Draft!}}
\author{Olivier Cailloux}
\author{Remzi Sanver}
\affil{Université Paris-Dauphine, PSL Research University, CNRS, LAMSADE, 75016 PARIS, FRANCE\\
	\href{mailto:olivier.cailloux@dauphine.fr}{olivier.cailloux@dauphine.fr}
}
\hypersetup{
	pdfsubject={},
	pdfkeywords={},
}

\begin{document}
\maketitle

\section{Arguments and DJ}
\label{sec:intro}
\begin{itemize}
	\item $A ≠ \emptyset$ is the set of arguments.
	\item $A^0 = \set{\emptyset}$, $A^1 = A$, and given $k \in \N^*$, $A^{k + 1} = A × A^k$. Throughout this article, $\N$ includes $0$ and $\N^* = \N \setminus \set{0}$.
	\item $\Astar = \bigcup_{k \in \N} A^k$ are all finite sequences of arguments, including $\emptyset$, the empty sequence of arguments.
	\item $\Phi = {\phi^+, \phi^−}$ are the possible judgments. The semantics of $\phi^−$ is to be the complement of $\phi^+$, e.g. $\phi^+$ is to be sure of something, and $\phi^−$ is to be doubtful or sure of the contrary). Given $\phi \in \Phi$, $\phibar$ denotes the element of $\Phi$ that is not $\phi$.
	\item $I ≠ \emptyset$ is the set of individuals.
	\item $\pib: I × \Astar → \Phi$ represents the reactions of individuals to sequences of arguments, called the behavioral function.
	\item Given $i \in I$, $\pib_i(\alpha) = \pib(i, \alpha)$, thus $\pib_i: \Astar → \Phi$ represents the reactions of $i$ to sequences of arguments.
\end{itemize}
\begin{remark}[$\pib$ and counterfactual behavioral functions]
	The symbol $\pib$ denotes a unique given function that is determined by the world we live in, and generally unknown to us. We will also use the symbol $\pi \in \Phi^{I × \Astar}$ to denote a generic behavioral function, that is, a generic function of the same form. This permits (when $\pi ≠ \pib$) counterfactual reasonings.
\end{remark}

$a$ is towards the end of $\alpha$ iff $a$ is the last or before last element of $\alpha$, thus, given $l$ the length of $\alpha$, iff $\exists \beta \suchthat \alpha = (\beta, a)$ or $\exists \beta, b \suchthat \alpha = (\beta, a, b)$.
\begin{definition}[Decisive argument]
	$a$ is decisive for $(\phi, i)$ iff $\forall \alpha \in \Astar$, if $a$ is towards the end of $\alpha$, then $\pib_i(\alpha) = \phi$.
\end{definition}

\begin{definition}[Deliberated judgment]
	$\phi$ is deliberated for $i$ iff $\exists a$ decisive for $(\phi, i)$.
\end{definition}
Let $\Phi_i$ denote the set of deliberated judgments of $i$: $\phi \in \Phi_i$ iff $\phi$ is deliberated for $i$.
\begin{remark}[Determined DJ VS DJ]
	To be more precise, we should write that $\phi \in \Phi_i$ means that $\phi$ is determined as being deliberated for i. And $\phi$ is determined as deliberated for $i$ implies that $\phi$ is deliberated for $i$.
\end{remark}
\begin{remark}[$i$ VS $\pib_i$]
	We can also replace $i$ with $\pib_i$ in those notations to make the relation to $\pib$ explicit, thus, we can write equivalently that $a$ is decisive for $(\phi, \pib_i)$, and that $\phi$ is deliberated for $\pib_i$, and that $\phi \in \Phi_{\pib_i}$. Further, these definitions naturally extend to apply to any $\pi$ instead of $\pib$, thus, we can write that $a$ is decisive for $(\phi, \pi_i)$, and so on.
\end{remark}

\begin{proposition}[Non contradiction]
	If $\phi \in \Phi_i$, then $\phibar \notin \Phi_i$.
\end{proposition}
\begin{proposition}[Incompleteness]
	$\exists \pi_i \suchthat \Phi_i = \emptyset$.
\end{proposition}

\section{Problem set up}
Finite observation never leads to knowing anything about DJs.
In other words, no verifiable hypothesis on the basis of finite observations are enough to know anything about the DJs. This is an instance of a well known problem: general enough theories can’t be finitely verified.
The usual “workaround” is to rely on falsifiable claims instead.
Capture of shallow (classical) preferences using only falsifiable claims was achieved by the celebrated approach proposed by Samuelson, relating preferences to choice behavior, which is observable.

In normative decision theories, the picture differs. Axioms are used that claim that we ought to decide in such and such a way, and these axioms are used to obtain conclusions on the basis of observable data. For example, the SEU principle is codified in axioms proposed by Savage. Observable data serve to parameterize the utility function of an individual, which can then be used for recommendation.
Normative axioms are neither verifiable, nor falsifiable. They simply rest on their own convincing power. This puts them out of reach of observations, and leads to endless disputes, as disagreements are not dissolvable by experiences. Also, their convincing power is usually accessible only to the specialists, who can read the formal language in which these axioms are phrased, and realize (some of the) consequences of accepting these axioms.
Relatedly, they cause a risk of being applied to recommend to persons who would not accept these axioms as codifying a reflexive behavior, would they fully understand these axioms and their consequences. 

We want to bridge the world of observable preferences, which permits to obtain results that rest on falsifiable or verifiable claims, and normative decision making, that wishes to be able to recommend courses of actions that may differ from the usual course of action that an individual would have taken without recommendation.

Our main task will be to study the ways we can deduce anything about anyone’s DJ on the basis of mostly verifiable or falsifiable claims. What general forms should these claims adopt?

%We can theorize on pi, this is falsifiable. We can mandate to predict the whole of pi, but it is not necessary. Thus, what form should such theory take? Characterize the class of falsifiable theories and represent them.
%Idea: if we are to know anything about the DJ with falsifiable hyp, we need to use my empirical theories of DJ.
%Thm [Debates suffice]:
%Thm [The necessity of debating]: the theory must have the form of a function that answers every argument.

\section{Observations and claims}
\begin{itemize}
	\item $Q \subseteq I × \Astar$ represents a set of queries.
	\item $\piQ \in \Phi^Q$ is the restriction of $\pi$ to the sub-domain $Q$, thus, $\piQ(i, \alpha) = \pi(i, \alpha), \forall (i, \alpha) \in Q$. The function $\pibQ$ represents a set of observations, and $\piQ$ is the set of observations corresponding to the query $Q$ under the hypothesis that $\pi$ would be the behavioral function.
	\item $\pi'$ is compatible with $\piQ$ iff $\restr{\pi'}{Q} = \piQ$.
	\item $\cpiQ$ is the set of behavioral functions compatible with the observations $\piQ$, thus $\pi' \in \cpiQ$ iff $\pi'$ is compatible with $\piQ$.
\end{itemize}

$\piQ$ permits to deduce that $i$ judges deliberately that $\phi$, denoted by $\piQ \vdash \phi \in \Phi_i$, iff $\forall \pi' \in \cpiQ: \phi \in \Phi_{\pi'_i}$.

\begin{example}[An anecdotal claim]
	\label{ex:anecdotal}
	Fix some $i$, $\phi$ and $a, b \in A$, and define $C = \set{\pi \suchthat \pi_i(a, b) = \phi}$.
	This claim permits to deduce nothing.
\end{example}
\begin{example}[A direct claim]
	\label{ex:direct}
	Fix some $i$ and $\phi$, and define $C = \set{\pi \suchthat \phi \in \Phi_i}$.
	This claim permits to deduce that $\phi \in \Phi_i$.
\end{example}
\begin{example}[A claim about decisiveness]
	\label{ex:decisiveness}
	Fix some $i$ and $\phi$, and define $C = \set{\pi \suchthat \exists a \suchthat a \text{ is decisive for } (\phi, \pi_i)}$. 
	This claim permits to deduce that $\phi \in \Phi_i$.
\end{example}
\begin{example}[A substantial claim about decisiveness]
	\label{ex:substantial}
	Fix some $i$, $\phi$ and $a$, and define $C = \set{\pi \suchthat a \text{ is decisive for } (\phi, \pi_i)}$.
	This claim permits to deduce that $\phi \in \Phi_i$.
\end{example}

\begin{proposition}[No finite observations suffice]
	$\nexists \text{ finite } \piQ, i, \phi \suchthat \piQ \vdash \phi \in \Phi_{\pi_i}$.
\end{proposition}

$C \subseteq \Phi^{I × \Astar}$ is a claim, it has the semantics: “$\pib \in C$”.

$C$ permits to deduce that $\phi \in \Phi_i$, denoted by $C \vdash \phi \in \Phi_i$ (using the deduction symbol $\vdash$ used in logic), iff $\forall \pi \in C: \phi \in \Phi_{\pi_i}$.

The following definition can be understood by realizing that, first, after having observed $\pibQ$, we know that $\pib \in \cpibQ$, and nothing more; and second, to prove that a claim $C$ holds, it is necessary and sufficient to prove that $\pib \in C$. Thus, intuitively, the observations $\pibQ$ prove the claim $C$ iff knowing only that $\pib \in \cpibQ$ is enough to conclude that $\pib \in C$.
\begin{definition}[Verification]
	\label{def:verif}
	A set of queries $Q$ verifies a claim $C$ iff $\forall \pi \in \cpibQ: \pi \in C$.
	A claim is \emph{verifiable} iff $\exists \text{ finite } Q$ that verifies it.
\end{definition}
\begin{definition}[Falsification]
	\label{def:fals}
	A claim $C$ is falsified by $\piQ$ iff $C \cap \cpiQ = \emptyset$. 
	A claim $C$ is \emph{falsifiable} iff $\exists \text{ finite } \piQ \suchthat C$ is falsified by $\piQ$.
\end{definition}
% [“No \pi' \in R complete \pi'|Q”, or “the set of claimed possibilities and the set of possibilities according to our observations do not cohere”].

\begin{example}[A verifiable claim]
	The claim defined in \cref{ex:anecdotal} is verifiable.
\end{example}
\begin{example}[A direct, non falsifiable claim]
	The claim defined in \cref{ex:direct} is not falsifiable.
\end{example}
\begin{example}[A claim that is falsifiable for finite arguments]
	The claim defined in \cref{ex:decisiveness} is falsifiable iff $A$ is finite.
\end{example}
\begin{remark}[Falsifiability under finite arguments]
	Being falsifiable under assumption of finiteness of $A$ may, from a cursory reading, be thought sufficient for practical applicability. Our perspective in this article differs. We grant that $A$ need not be infinite, but in most non trivial applications, it will be large. If a claim is not falsifiable under infinite arguments but is falsifiable under finite arguments, then falsifying that claim might reveal extremely hard: as a general reasoning will not suffice to provide falsification, some reasoning that examines each argument in turn might be required.
	
	A case in point is \cref{ex:decisiveness}: falsifying that claim requires to show experimentally that no argument is decisive. This basically reverses the burden of the proof from the claimant to the person who attempts to falsify the claim.
	(Obviously, the very falsificationist approach inevitably involves such an inversion of the burden of the proof, but reasonable efforts must be made to ease falsification if this approach is not to turn into a blanket permission for asserting claims that are simply very hard to disprove.)
	
	Therefore, we think it is reasonable to mandate that claims be falsifiable also when $A$ is infinite. In other words, we consider “infinity” as a mere extreme case of being “large enough” that systematic examination of the arguments one by one be not feasible for practical reasons. It is of course possible to obtain more precise results involving the cardinality of the considered sets, but at this stage of the investigation, we think that such an attempt would obscure our results for a gain in precision that risks to be seldom useful.
\end{remark}
	
\begin{example}[A falsifiable, non verifiable claim]
	The claim defined in \cref{ex:substantial} is falsifiable but is not verifiable.
\end{example}

\begin{proposition}[No verifiable claim suffices]
	$\nexists \text{ verifiable } C, i, \phi \suchthat C \vdash \phi \in \Phi_i$.
\end{proposition}

\begin{proposition}[Simple observability]
	\label{prop:simpleobs}
	$\exists \text{ falsifiable } C \suchthat C \vdash \phi \in \Phi_i$.
\end{proposition}
\begin{proof}
	\Cref{ex:substantial} provides a suitable example.
\end{proof}
\Cref{prop:simpleobs} provides a positive result, but it may reveal non applicable in many non-trivial situations, for multiple reasons. Not every observations may be possible. Individuals may not all have the same DJ. When individuals have the same DJ, they may not accept the same decisive argument: depending on their background, different argumentative paths may be required to lead them to their DJ. Even when individuals accept the same decisive argument, this argument may be unknown to the scientist, thus obtaining a claim of the form exhibited in \cref{ex:substantial} may be unduly demanding. We want to explore the possibility (or necessity, depending on the circumstances) of using weaker claims.

Let $l_\alpha$ denote the length of the sequence of arguments $\alpha$. Say that $\alpha$ is a prefix of $\beta$ iff $l_\alpha ≤ l_\beta$ and $\forall 1 ≤ k ≤ l: \alpha_k = \beta_k$.

\begin{definition}[Short memory]
	$Q$ assumes short memory iff $\exists i, \alpha, \alpha' \suchthat (i, \alpha), (i, \alpha') \in Q$, with neither of $\alpha$ and $\alpha'$ being a prefix of the other one.
\end{definition}
Queries $Q$ are said to be memory-compatible otherwise, i.\ e., when $Q$ interrogates each $i$ only about sequences of arguments that can be arranged in such a way that one is a prefix of the previous one (except for the first one).
A claim is memory-compatibly falsifiable iff $C$ is falsified by some finite $\piQ$ with $Q$ being memory-compatible.
Observe that \cref{prop:simpleobs} fails when mandating memory-compatibility.
\begin{proposition}[No memory-compatible observability for finite $I$]
	Assuming a finite $I$ and infinitely many arguments, $\nexists \text{ memory-compatibly falsifiable } C, i, \phi \suchthat C \vdash \phi \in \Phi_i$.
\end{proposition}

\begin{theorem}[Observability of DJ]
	Assuming infinite $I$, $\exists \text{ memory-compatibly}$ $\text{ falsifiable } C, i, \phi \suchthat C \vdash \phi \in \Phi_i$.
\end{theorem}

\section{Querying strategies}
We could also explore the realm of querying strategies: how should we query individuals to obtain (partial) knowledge of their DJ?

\begin{definition}[Querying strategy]
	A querying strategy $(Q, q)$ is a pair composed of a set $Q$ of queries and a function $q: \Phi^Q → I × \Astar$ that indicates which question should be asked next depending on the previous results.
\end{definition}

\appendix
\section{Falsifiability}
$C$ is tautological iff $C = \Phi^{I × Astar}$.
Otherwise, intuition suggests that if it is verifiable, then it is falsifiable (“all verification attempts could have failed”).

%\bibliography{bibl}

\end{document}

